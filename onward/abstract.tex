\begin{abstract}
	Testing is perhaps the most popular way of ensuring software quality. Despite that, there are few principled methods to automatically test software, delta debugging being a notable exception. The gap that we address in this paper is how to efficiently test a software system under black-box assumptions with clearly articulated formal guarantees, where by efficiency we mean that only a small proportion of the overall test inputs is tried by the testing algorithm, and by formal guarantees we mean that the number of tests and probability of success follow from the testing framework.
	
	We address this challenge by making novel use of a technique from the area of signal processing known as \emph{compressed sensing}, and in particular its boolean variant known as \emph{group testing}. The key idea is to identify tests of interest by testing the system with random inputs that capture its input/output behavior, where group testing prescribes that tests are grouped together to minimize the number of inputs with a mathematical procedure for how to recover single-element information from the results w.r.t. groups.
	
	As preliminary validation of the efficacy of compressed sensing in software testing, we report on experiments with web sanitizers and validators. The test inputs are cross-site scripting (XSS) payloads, and the output is a boolean indication whether or not the payloads penetrated through the built-in defenses. We demonstrate our ability to crack such defenses using only on the order of $K \cdot \log (N)$ tests, where $N$ is the number of single elements, whereas brute-force testing would require $N$ tests. $K<<N$ in our framework, and with $N$ in the thousands, this marks a large reduction in computation. 
\end{abstract}
